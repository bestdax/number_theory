\chapter{连分数}
\section{什么是连分数}\label{sec:连分数介绍}
连分数是一个很有用的工具。我们先来举几个例子,说明什么叫连分数以及它的用处.
\begin{example}\label{ex:开根号}
	如果你手边没有平方根表,也没有计算器,那么能用 什么简单方法来求 \( \sqrt{11} \)的近似值?当然可以用通常的求平方根的
	方法,但下面的办法看来更方便。
	\begin{align}
		             & 3 < \sqrt{11} <4,                                                 \\
		\sqrt{11}  = & 3 +(\sqrt{11}-3)                                   \label{cf2}  \\
		=            & 3 + \frac{1}{(\sqrt{11}+3)/2} \nonumber                           \\
		=            & 3 + \cfrac{1}{3+(\sqrt{11}-3)/2}                      \label{cf3} \\
		=            & 3 + \cfrac{1}{3 + \cfrac{1}{\sqrt{11} + 3}} \nonumber             \\
		=            & 3 + \cfrac{1}{3 + \cfrac{1}{6 + (\sqrt{11} -3)}}.\label{cf4}
	\end{align}

	重复这一过程就可得到
	\begin{align}
		\sqrt{11} & =
		3 + \cfrac{1}{3 +
			\cfrac{1}{6 +
				\cfrac{1}{3 +
					(\sqrt{11} -3)/2
		}}}          \label{cf5} \\
		          & =
		3 + \cfrac{1}{3 +
			\cfrac{1}{6 +
				\cfrac{1}{3 +
					\cfrac{1}{6 +
						(\sqrt{11} -3)
					}}}}          \label{cf6}
	\end{align}
	\begin{align}
		 & =
		3 + \cfrac{1}{3 +
			\cfrac{1}{6 +
				\cfrac{1}{3 +
					\cfrac{1}{6 +
						\cfrac{1}{3 +
							(\sqrt{11} -3) / 2
		}}}}}        \label{cf7}  \\
		 & =
		3 + \cfrac{1}{3 +
			\cfrac{1}{6 +
				\cfrac{1}{3 +
					\cfrac{1}{6 +
						\cfrac{1}{3 +
							\cfrac{1}{6 +
								(\sqrt{11} -3)
		}}}}}}        \label{cf8} \\
		= \dots\dots. \nonumber
	\end{align}
	马上就会想到分别把式\eqrefl{cf2,cf3,cf4,cf5,cf6,cf7,cf8}中的无理数 \( (\sqrt{11} - 3) \) 或 \( (\sqrt{11} - 3)/2
	\)去掉后所得到的\enquote{分数}值来作为 \( \sqrt{11} \)的\enquote{近似值}。容易算出,这些\enquote{近似值}依次为:
	\begin{equation}\label{eq:根号11}
		\sqrt{11} \approx 3, \frac{10}{3}, \frac{63}{19}, \frac{199}{60}, \frac{1257}{379}, \frac{3970}{1197},
		\frac{25077}{7561}.
	\end{equation}
	用小数表示,这些\enquote{近似值}依次为(取八位小数):
	\begin{equation}
		\sqrt{11} \approx 3,\: 3.33333333,\:3.31578947,\:3.316666666,\: 3.316622691,\: 3.31662489,\: 3.31662478.
	\end{equation}
	这些的确是很精确的近似值,因为实际上
	\begin{equation}
		\sqrt{11} = 3.31662479\dots.
	\end{equation}
	若取 \( 10 / 3 \)作近似值,就精确到 \( 2 / 10^2 \);取 \( 63 / 19 \)就精确到 \( 9 / 10^4 \);取 \( 199 / 60 \)就精确到
	\( 42 / 10^6 \);取 \( 1257 / 379 \)就精确到 \( 22 / 10^7 \);取 \( 3970 / 1197 \)就精确到 \( 1 / 10^7 \);取 \( 25077 /
	7561\)就精确到 \( 1/10^8 \)。这些数据表明,这些近似值依次一个比一个更精确。此外,容易看出
	\begin{equation}
		3 < \frac{63}{19} < \frac{1257}{379} < \frac{25077}{7561} < \sqrt{11} < \frac{3970}{1197} < \frac{199}{60} <
		\frac{10}{3}.
	\end{equation}
	这个例子表明,它提出了一方法来构造一些特殊形式的 \enquote{分数},作为无理数的近似值。因而也就提出了研究这种形式
	\enquote{分数}的性质的新课题,以及从理论上研究无理数的这种形式的有理数逼近。另外,以上的过程不断的继续下去,就可以得到
	一个无穷尽的\enquote{分数}表达式:
	\begin{equation}
		3 + \cfrac{1}{3 +
			\cfrac{1}{6 +
				\cfrac{1}{3 +
					\cfrac{1}{6 +
						\cfrac{1}{3 +
							\cfrac{1}{6 +
								\cfrac{1}{3 +
									\cfrac{1}{6 +
										\genfrac{}{}{0pt}{}{\phantom{1}}{\ddots}
									}}}}}}}}
	\end{equation}
	这种表达式的确切含意是什么呢?能否定义它的\enquote{值}?如果能定义它的\enquote{值},那么这\enquote{值}和 \( \sqrt{11}
	\)有什么关系?这就又提出了进一步的硏究课题。
\end{example}

\begin{example}\label{ex:化简分数}
	一个分母、分子很大的分数用起来是很不方便的,如 \( 103993/33102 \)。 我们想找一个分母、分子较小的分数来近似它,
	希望分母不要太大,但误差很小。利用例\ref{ex:开根号}的方法可得
	\begin{align*}
		\frac{103993}{33102} & = 3 + \frac{4687}{33102} = 3 + \cfrac{1}{7 + \cfrac{293}{4687}}     \\
		                     & = 3 + \cfrac{1}{7  + \cfrac{1}{15 + \cfrac{292}{293}}}              \\
		                     & = 3 + \cfrac{1}{7  + \cfrac{1}{15 + \cfrac{1}{1 + \cfrac{1}{292}}}}
	\end{align*}
	类似于例\ref{ex:开根号},我们扔掉这些\enquote{分数}中小于1的数 \(\displaystyle \frac{4687}{33102},\,\frac{293}{4687},\,
	\frac{292}{293},\, \frac{1}{292}, \)用依次得到的
	\begin{equation*}
		3,\ 3 + \frac{1}{7} = \frac{22}{7},\ 3+ \cfrac{1}{7 + {\cfrac{1}{15}}} = \frac{333}{106},\ 3 + \cfrac{1}{7 +
			\cfrac{1}{15 + \cfrac{1}{1}}} = \frac{355}{113}
	\end{equation*}
	来近似 \( 103993/33102 = 3.141592653\dots \)。由
	\begin{equation*}
		\frac{22}{7} = 3.14285714\dots,\ \frac{333}{106} = 3.14150943\dots,\ \frac{355}{113} = 3.14159292\dots,
	\end{equation*}
	推出它们的精确度依次为 \(  14/10^2,\, 13/10^4,\, 8/10^5,\, 3/10^7 \)。与它们的分母相比(依次为: \( 1,7,106,113
	\))精确度是很高的。事实上,这些都是圆周率\pi 的近似值, \( 22/7 \)是所谓\enquote{疏率}, \( 355/113 \)是\enquote{密率}。

	这个例子表明,即使是一个分数把它表成这种形式的\enquote{分数} 也是有好处的。
\end{example}

\begin{example}
	至今,除了试算具体数值,我们还没有方法来求解不定方程
	\begin{equation}
		x^2 - 11y^2 = 1, \qquad x > 0, y > 0. \label{eq:不定方程}
	\end{equation}
	这个不定方程可化为
	\begin{align}
		x - \sqrt{11}y          & = \frac{1}{x + \sqrt{11}y}, x > 0, y > 0. \nonumber \\
		\frac{x}{y} - \sqrt{11} & = \frac{1}{y(x + \sqrt{11}y)}, x > 0, y > 0.
	\end{align}
	这表明不定方程\eqref{eq:不定方程}的解 \( x, y \)所给出的分数 \( x/y \)是 \( \sqrt{11}
	\)一个很精确的近似值。因而,我们可以试想从式\eqref{eq:根号11}所得的那些 \( \sqrt{11}
	\)的近似分数中去寻找\eqref{eq:不定方程}的解。通过验算知,由分数 \( 199/60, 3970/1197 \)得出的
	\begin{equation}
		x=199,\quad y=60; \quad x=3970, \quad y=1197
	\end{equation}
	是\eqref{eq:不定方程}的解,其它几个都不是。当然,从式\eqref{cf8}继续算下去得到的近似分数中还能找出解来。

	这个例子启示我们,通过研究这类新形式的\enquote{分数},有可能找到求解形如\eqref{eq:不定方程}的一类不定方程的方法。
\end{example}
\begin{definition}
	设 \( x_0, x_1, x_2, \dots \) 是一个无穷实数列, \( x_j > 0, j \geqslant 1 \)。 对缩写的 \( n \geqslant 0 \), 我们把表示式
	\begin{equation}\label{eq:有限连分数定义}
		x_0 +
		\cfrac{1}{x_1 +
			\cfrac{1}{x_2 +
				\cfrac{1}{x_3 +
					\cfrac{1}{\ddots +
						\cfrac{1}{x_n}
					}}}}
	\end{equation}
	称为\textbf{( \( n \) 阶)有限连分数},它的值是一个实数。当 \( x_0, \dots, x_n
	\)均为整数时称为\textbf{( \( n \)阶) 有限简单连分数},它的值是一个有理分数。为书写方便,把有限连分数记作
	\begin{equation}
		\langle x_0,x_1,\dots,x_n\rangle.
		\label{eq:有限连分数定义简}
	\end{equation}
	设 \( 0 \leqslant k \leqslant n \) , 我们把有限连分数
	\begin{equation}
		\langle x_0, x_1, \dots, x_k\rangle
		\label{eq:第k个渐近分数}
	\end{equation}
	称为是有限连分数\eqref{eq:有限连分数定义简}的\textbf{第 \( k \)个渐进分数}。当\eqref{eq:有限连分数定义简}是有限简单连分数(即 \( x_0,\dots,x_n
	\)均为整数)时,把 \( x_k(0 \leqslant k \leqslant n) \) 称为是它的\textbf{第 \( k \)
		个部分商}。当式\eqref{eq:有限连分数定义}(或\eqref{eq:有限连分数定义简})中的 \( n \to \infty
	\)时,我们把相应的表示式\eqref{eq:有限连分数定义}(或\eqref{eq:有限连分数定义简})称为\textbf{无限连分数},即表示式
	\begin{equation}\label{eq:无限连分数定义}
		x_0 + \cfrac{1}{x_1 +
			\cfrac{1}{x_2 +
				\cfrac{1}{x_3 +
					\genfrac{}{}{0pt}{}{\vphantom{1}}{\ddots}
				}}},
	\end{equation}
	或者简记为
	\begin{equation}
		\langle x_0, x_1, x_2, \dots\rangle.
		\label{eq:无限连分数定义简}
	\end{equation}
	当 \( x_j(j \geqslant 0) \)均为整数时,称\eqref{eq:无限连分数定义}(或\eqref{eq:无限连分数定义简})为\textbf{无限简单连分数}。同样的,对任意
	\( k \geqslant 0
	\),有限连分数\eqref{eq:有限连分数定义简}称为是无限连分数\eqref{eq:无限连分数定义}(或\eqref{eq:无限连分数定义简})的\textbf{第 \( k \)个渐近分数};当\eqref{eq:无限连分数定义}(或\eqref{eq:无限连分数定义简})是无限简单连分数时, \( x_k(k \geqslant 0)
	\)称为是它的\textbf{第 \( k \)个部分商}。如果存在极限
	\begin{equation}
		\lim_{k\to\infty}\langle x_0,\dots,x_k\rangle = \theta,
		\label{eq:无限连分数收敛}
	\end{equation}
	那么,就说\textbf{无限连分数}\eqref{eq:无限连分数定义}(或\eqref{eq:无限连分数定义简})\textbf{是收敛的}, \( \theta \)
	称为是\textbf{无限连分数\eqref{eq:无限连分数定义}(或\eqref{eq:无限连分数定义简})的值},记作
	\begin{equation}
		\langle x_0,x_1,x_2,\dots\rangle = \theta,
		\label{eq:无限连分数值}
	\end{equation}
	或极限\eqref{eq:无限连分数收敛}不存在,则说\textbf{无限连分数}\eqref{eq:无限连分数定义}(或\eqref{eq:无限连分数定义简})\textbf{是发散的}。
\end{definition}
本章主要讨论简单连分数的基本理论及其应用。作为本节的结束,我们来证明有限连分数的一些最基本的性质,这在以后是经常要用的。
\begin{theorem}\label{thrm:大小}
	设 \( x_0, x_1, x_2, \dots \)是无穷实数列, \( x_j > 0, j \geqslant 1 \)。那么,
	\begin{enumerate}[label=\rm(\roman*)]
		\item 对任意的整数 \( n \geqslant 1, r \geqslant 1 \)有
		      \begin{flalign}
			      \langle x_0,\dots,x_{n-1},x_n,\dots,x_{n+r}\rangle \nonumber &                                                                                        \\
			                                                                   & = \langle x_0,\dots,x_{n-1},\langle x_n,\dots,x_{n+r}\rangle\rangle\nonumber           \\
			                                                                   & = \langle x_0,\dots,x_{n-1}, x_{n} + 1 / \langle x_{n+1},\dots,x_{n+r}\rangle\rangle .
			      \label{thrm01eq:01}
		      \end{flalign}
		      特别地有
		      \begin{equation}
			      \langle x_0, \dots, x_{n-1},x_n,x_{n+1}\rangle=\langle x_0,\dots,x_{n-1},x_n+1/x_{n+1}\rangle.
			      \label{thrm01eq:02}
		      \end{equation}
		\item 对任意实数 \( \eta>0 \)及整数 \( n \geqslant 1 \),
		      \begin{gather}
			      \langle x_0,\dots, x_{n-1}, x_n\rangle > \langle x_0,\dots,x_{n-1}, x_{n}+\eta\rangle, 2 \nmid n.
			      \label{thrm01eq:03} \\
			      \langle x_0,\dots, x_{n-1}, x_n\rangle < \langle x_0,\dots,x_{n-1},x_n + \eta\rangle, 2 \mid n.
			      \label{thrm01eq:04}
		      \end{gather}
		\item 记
		      \begin{equation}\label{eq:an标记}
			      a^{(n)} = \langle x_0, \dots, x_n \rangle.
		      \end{equation}
		      我们有
		      \begin{gather}
			      a^{(n)} > a^{(n + r)}, 2 \nmid n, r \geqslant 1, \label{thrm01eq:05}\\
			      a^{(n)} < a^{(n + r)}, 2 \mid n, r \geqslant 1, \label{thrm01eq:06}\\
			      a^{(1)} > a^{(3)} > a^{(5)} > \dots > a^{2s - 1} > \dots, \label{thrm01eq:07}\\
			      a^{(0)} < a^{(2)} < a^{(4)} < \dots < a^{2s} < \dots, \label{thrm01eq:08}\\
			      a^{(2s-1)} > a^{(2t)}, s \geqslant 1, t \geqslant 0 \label{thrm01eq:09}.
		      \end{gather}
	\end{enumerate}

\end{theorem}
\begin{proof}
	式\eqref{thrm01eq:01}和\eqref{thrm01eq:02}直接由有限连分数的定义式\eqref{eq:有限连分数定义}和\eqref{eq:有限连分数定义简}推出。对
	\( n \) 用归纳法来证式\eqref{thrm01eq:03}和\eqref{thrm01eq:04},当 \( n=1 \)和 \( n=2
	\)时,式\eqref{thrm01eq:03}和\eqref{thrm01eq:04}显然成立。假设当 \( n=2k-1 \)和 \( n=2k (k \geqslant
	1)\)时式\eqref{thrm01eq:03}和\eqref{thrm01eq:04}都成立。当 \( n=2(k+1)-1=2k+1 \)时,由式\eqref{thrm01eq:01}知
	\begin{equation*}
		\langle x_0, \dots, x_{2k+1} \rangle = \langle x_0, x_1, \langle x_2, \dots, x_{2k+1} \rangle\rangle.
	\end{equation*}
	由假设式\eqref{thrm01eq:03}对 \( n = 2k -1 \)成立,所以
	\begin{equation*}
		\langle x_2, \dots, x_{2k}, x_{2k+1} \rangle > \langle x_2, \dots, x_{2k}, x_{2k+1} + \eta \rangle.
	\end{equation*}
	由上式及 \( n=2 \)时式\eqref{thrm01eq:04}成立就推出
	\begin{equation*}
		\langle x_0, x_1, \langle x_2, \dots, x_{2k}, x_{2k+1} + \eta \rangle\rangle < \langle x_0, x_1, \langle
		x_2, \dots, x_{2k}, x_{2k+1} \rangle\rangle,
	\end{equation*}
	由此及式\eqref{thrm01eq:01}就推出当 \( n=2(k+1) - 1 \)时式\eqref{thrm01eq:03}成立。当 \( n=2(k+1) = 2k + 2
	\)时,由式\eqref{thrm01eq:01}知
	\begin{equation*}
		\langle x_0, \dots, x_{2k+2} \rangle = \langle x_0, x_1, \langle x_2, \dots, x_{2k+2}\rangle\rangle.
	\end{equation*}
	由假设式\eqref{thrm01eq:04}对 \( n=2k \)成立,所以
	\begin{equation*}
		\langle x_2, \dots, x_{2k+1}, x_{2k+2} \rangle < \langle x_2, \dots, x_{2k+1}, x_{2k+2} + \eta \rangle.
	\end{equation*}
	由上式及 \( n=2 \)时式\eqref{thrm01eq:04}成立就推出
	\begin{equation*}
		\langle x_0, x_1, \langle x_2, \dots, x_{2k+1}, x_{2k+2} \rangle\rangle < \langle x_0, x_1, \langle x_2,
		\dots, x_{2k+1}, x_{2k+2} + \eta\rangle\rangle.
	\end{equation*}
	由此及式\eqref{thrm01eq:01}就推出当 \( n=2(k+1)
	\)时式\eqref{thrm01eq:04}成立,这就证明了式\eqref{thrm01eq:03}和\eqref{thrm01eq:04}对所有的 \( n \)都成立。
	由式\eqref{thrm01eq:01}知
	\begin{equation*}
		a^{(n+r)} = \langle x_0, \dots, x_{n-1}, x_{n} + 1/\langle x_{n+1}, \dots, x_{n+r}\rangle\rangle.
	\end{equation*}
	由此及式\eqref{thrm01eq:03}和\eqref{thrm01eq:04}就分别推出式\eqref{thrm01eq:05}和\eqref{thrm01eq:06}。取 \( n=1,
	r=2,4,6,\dots \),从式\eqref{thrm01eq:05}就推出式\eqref{thrm01eq:07}。取 \( n=0, r=2,4,6,\dots
	\),从式\eqref{thrm01eq:06}就推出式\eqref{thrm01eq:08}。最后由式\eqref{thrm01eq:05}及式\eqref{thrm01eq:06}得
	\begin{equation*}
		a^{2s-1} > a^{2s-1 + 2t + 1} = a^{(2s+2t)}  > a^{(2t)},
	\end{equation*}
	这就证明了式\eqref{thrm01eq:09}。\footnote{事实上,(ii)和(iii)都可由以下简单结论直接看出:一个分数 \( a / b \)
		( \( a, b \)是正实数),当分母 \( b \)变大时变小,当分母 \( b \)变小时变大。具体的证明请读者给出。}
\end{proof}
下面要证明的性质是:如何给出一个明确的公式,用 \( x_0, x_1, \dots, x_n \)来表示连分数 \( \langle x_0, \dots, x_n \rangle
\)的值。先来考虑几个具体表达式。
\begin{align*}
	\langle x_0 \rangle                & = \frac{x_0}{1},                                                              \\
	\qquad \langle x_0, x_1 \rangle    & = x_0 + \frac{1}{x_1} = \frac{x_0x_1+1}{x_1},                                 \\
	\langle x_0, x_1, x_2 \rangle      & = \langle x_0, x_1 + 1/x_2 \rangle = \frac{x_0(x_1 + 1/x_2) + 1}{x_1 + 1/x_2} \\
	                                   & = \frac{(x_0x_1 + 1)x_2 + x_0}{x_1 x_2 + 1}                                   \\
	\langle x_0, x_1, x_2, x_3 \rangle & = \langle x_0, x_1, x_2 +  1/x_3 \rangle                                      \\
	                                   & = \frac{(x_0 x_1 + 1)(x_2 + 1 / x_3) + x_0}{x_1(x_2 + 1/x_3) + 1}             \\
	                                   & = \frac{((x_0 x_1)x_2 + x_0)x_3 + (x_0 x_1 + 1)}{(x_1 x_2 + 1)x_3 + x_1}.
\end{align*}
因此,如果把 \( x_0, x_1, \dots \)看作是实变数,那么,
\begin{equation}\label{eq:渐进分数}
	\langle x_0, \dots, x_{n-1}, x_n \rangle = P_n / Q_n, \qquad n \geqslant 0,
\end{equation}
其中
\begin{equation}\label{eq:递推函数形式}
	P_n = P_n(x_0, \dots, x_n), \qquad Q_n = Q_n(x_0, \dots, x_n)
\end{equation}
是变数 \( x_0, \dots, x_n \)的整系数多项式(事实上 \( Q_n \)与 \( x_0
\)无关),且对每个变数的方次至多为一次(为什么)。故而也可这样表示:
\begin{equation}\label{eq:递推原形}
	\langle x_0, \dots, x_{n-1}, x_n \rangle = \frac{K_{n-1}x_n + D_{n-1}}{H_{n-1}x_n + E_{n-1}}, \qquad n \geqslant 1,
\end{equation}
其中
\begin{align*}
	K_{n-1} = K_{n-1}(x_0, \dots, x_{n-1}),\qquad & H_{n-1} = H_{n-1} (x_0, \dots, x_{n-1}), \\
	D_{n-1} = D_{n-1}(x_0, \dots, x_{n-1}),\qquad & E_{n-1} = E_{n-1} (x_0, \dots, x_{n-1}), \\
\end{align*}
都是变数 \( x_0, \dots, x_{n-1} \)的整系数多项式。在式\eqref{eq:递推原形}中,保持 \( x_0, \dots, x_{n-1} \)不变。令 \(
x_n \to + \infty \)时,
\begin{gather*}
	\langle x_0, \dots, x_{n-1}, x_n \rangle \to \langle x_0, \dots, x_{n-1} \rangle = \frac{P_{n-1}}{Q_{n-1}}. \\
	\frac{K_{n-1}x_n + D_{n-1}}{H_{n-1}x_n + E_{n-1}} \to \frac{K_{n-1}}{H_{n-1}}
\end{gather*}
令 \( x_n \to 0 \)时,
\begin{gather*}
	\langle x_0, \dots, x_{n-1}, x_n \rangle \to \langle x_0, \dots, x_{n-2} \rangle = \frac{P_{n-2}} {Q_{n-2}}, \\
	\frac{K_{n-1}x_n + D_{n-1}}{H_{n-1}x_n + E_{n-1}} \to \frac{D_{n-1}}{E_{n-1}}
\end{gather*}
以上关系式告诉我们应该有
\begin{equation*}
	K_{n-1} = P_{n-1}, \quad H_{n-1} = Q_{n-1}, \quad D_{n-1} = P_{n-2}, \quad E_{n-1} = Q_{n-2}.
\end{equation*}
由此及式\eqrefl{eq:渐进分数,eq:递推原形}进一步推测应有递推关系式:
\begin{align}\label{eq:递推}
	\begin{cases}
		P_n  = x_nP_{n-1} + P_{n-2}, \\
		Q_n  = x_nQ_{n-1} + Q_{n-2}.
	\end{cases}
\end{align}

\begin{theorem} \label{thrm:PQ}
	设 \( x_0, x_1, x_2, \dots \)是无穷实数列, \( x_j > 0, j \geqslant 1 \)。再设
	\begin{equation}\label{thrm02eq:00}
		P_{-2} = 0,\qquad P_{-1} = 1,\qquad Q_{-2} = 1,\qquad Q_{-1} = 0,
	\end{equation}
	以及当 \( n \geqslant 0 \)时, \( P_n, Q_n \)由递推关系公式\eqref{eq:递推}给出。那么,
	\begin{align}
		\langle x_0, \dots, x_n \rangle & = P_n / Q_n,  & n \geqslant \phantom{-}0. \label{thrm02eq:01} \\
		P_nQ_{n-1} - P_{n-1}Q_n         & = (-1)^{n+1}, & n \geqslant -1,           \label{thrm02eq:02} \\
		P_nQ_{n-2} - P_{n-2}Q_n         & = (-1)^nx_n,  & n \geqslant \phantom{-}0,\label{thrm02eq:03}
	\end{align}
	以及
	\begin{align}
		\langle x_0, \dots, x_{n-1}, x_n \rangle - \langle x_0, \dots, x_{n-1} \rangle
		 & = (-1)^{n+1}(Q_nQ_{n-1})^{-1}, & n \geqslant 1, \label{thrm02eq:04} \\
		\langle x_0, \dots,x_{n-2},  x_{n-1}, x_n \rangle - \langle x_0, \dots, x_{n-2} \rangle
		 & = (-1)^{n}(Q_nQ_{n-2})^{-1},   & n \geqslant 2.\label{thrm02eq:05}
	\end{align}
\end{theorem}

\begin{proof}
	当 \( n = 0 \) 时, \( P_0 = x_0,\, Q_0 = 1 \), 所以式\eqref{thrm02eq:01}成立。假设当 \( n = k(\geqslant 0) \)时\eqref{thrm02eq:01}成立。当 \( n = k + 1 \) 时,由式\eqref{thrm01eq:02}得
	\begin{equation*}
		\langle x_0, \dots, x_{k-1}, x_k, x_{k+1} \rangle = \langle x_0, \dots, x_{k-1}, x_k + 1/x_{k+1} \rangle,
	\end{equation*}
	由假设当 \( n = k \) 时式 \eqref{thrm02eq:01}成立及式\eqref{eq:递推},就推出
	\begin{align*}
		\langle x_0, \dots, x_{k-1}, x_k, x_{k+1} \rangle
		 & = \frac{(x_k + 1/x_{k+1})P_{k-1}+P_{k-2}}{(x_k + 1/x_{k+1})Q_{k-1} + Q_{k-2}}           \\
		 & = \frac{x_{k+1}(x_kP_{k-1} + P_{k-2})+P_{k-1}}{x_{k+1}(x_kQ_{k-1} + Q_{k-2}) + Q_{k-1}} \\
		 & = \frac{x_{k+1}P_k + P_{k-1}}{x_{k+1}Q_k + Q_{k-1}}                                     \\
		 & = \frac{P_{k+1}}{Q_{k+1}}
	\end{align*}
	即当 \( n = k + 1 \)时式\eqref{thrm02eq:01}也成立。所以,式\eqref{thrm02eq:01}当 \( n \geqslant 0 \)时都成立。

	当 \( n = -1 \)时,由式\eqref{thrm02eq:00}推出\eqref{thrm02eq:02}成立。当 \( n \geqslant 0 \)时,由式\eqref{eq:递推}可得 (消去 \( x_n \))
	\begin{equation}
		P_nQ_{n-1} - P_{n-1}Q_n = -(P_{n-1}Q_{n-2} - P_{n-2}Q_{n-1}).
	\end{equation}
	反复利用上式就推出
	\begin{equation*}
		P_nQ_{n-1} - P_{n-1}Q_n = (-1)^{n+1}(P_{-1}Q_{-2} - P_{-2}Q_{-1}).
	\end{equation*}
	由此及式\eqref{thrm02eq:00}就得到式\eqref{thrm02eq:02}。注意到当 \( n \geqslant 0 \)时 \( Q_n > 0 \),
	由式\eqref{thrm02eq:01}及\eqref{thrm02eq:02},就推出式\eqref{thrm02eq:04}。
	当 \( n \geqslant 0 \)时,由式\eqref{eq:递推}可得
	\begin{equation}
		P_nQ_{n-2} - P_{n-2}Q_n = (P_{n-1}Q_{n-2} - P_{n-2}Q_{n-1})x_n.
	\end{equation}
	由此及式\eqref{thrm02eq:02}就证明了式\eqref{thrm02eq:03}。注意到 \( n \geqslant 0 \)时 \( Q_n > 0
	\),由式\eqref{thrm02eq:01}及\eqref{thrm02eq:02}就推出式\eqref{thrm02eq:05}
\end{proof}
利用定理\ref{thrm:PQ}很容易出出定理\ref{thrm:大小}的(iii),即式\eqref{thrm01eq:05}---\eqref{thrm01eq:09}成立。详细推导留给读者。当然,那里的证明更简单。

下面来举几个例子。
\begin{example}
	求有限连分数 \( \langle -2, 1, 2/3, 2, 1/2, 3 \rangle \)的值。
\end{example}
\begin{solution}
	我们利用式\eqref{thrm01eq:02}来计算。
	\begin{align*}
		  & \langle -2, 1, 2/3, 2, 1/2, 3 \rangle & = & \langle -2, 1, 2/3, 2, 1/2 + 1/3 \rangle \\
		= & \langle -2, 1, 2/3, 2, 5/6 \rangle    & = & \langle -2, 1, 2/3, 2 + 6/5 \rangle      \\
		= & \langle -2, 1, 2/3, 16/5 \rangle      & = & \langle -2, 1, 2/3 + 5/16 \rangle        \\
		= & \langle -2, 1, 47/48 \rangle          & = & \langle -2, 1 + 48/47 \rangle            \\
		= & \langle -2, 95/47 \rangle             & = & -2 + 47/95 = -143/95.
	\end{align*}
\end{solution}

\begin{example}
	求有限简单连分数 \( \langle 1,1,1,1,1,1,1,1,1,1 \rangle \)的各个渐近分数。
\end{example}
\begin{solution}
	当然,我们可以用例\ref{ex:开根号}的方法一个一个计算,但这时利用定理\ref{thrm:PQ}递推地计算出 \( P_n,Q_n
	\)比较方便。按公式\eqref{thrm02eq:00},\eqref{eq:递推}可列
	出下表,这里 \( x_n=1,0 \leqslant 1 \leqslant 9 \),以及 \( P_{-2}=0,P_{-1}=1,Q_{-2}=1,Q_{-1} = 0 \),
	\begin{equation*}
		P_n = P_{n-1} + P_{n-2}, Q_n = Q_{n-1} + Q_{n-2}, n \geqslant 0.
	\end{equation*}

	\begin{table}[htbp]
		\centering
		\begin{tabularx}{\textwidth}{XXXXXXXXXXX}\toprule
			\( n \)   & 0 & 1 & 2 & 3 & 4 & 5  & 6  & 7  & 8  & 9  \\ \midrule
			\( x_n \) & 1 & 1 & 1 & 1 & 1 & 1  & 1  & 1  & 1  & 1  \\
			\( P_n \) & 1 & 2 & 3 & 5 & 8 & 13 & 21 & 34 & 55 & 89 \\
			\( Q_n \) & 1 & 1 & 2 & 3 & 5 & 8  & 13 & 21 & 34 & 55 \\ \bottomrule
		\end{tabularx}
	\end{table}
	因此,各个渐近分数 \( P_n/Qn \)依次为 \( 1/1, 2/1, 3/2, 5/3, 8/5, 13/8, 21/31, 34/21, 55/34, 89/55 \)。显见, \( P_n, Q_n
	\)均为Fibonacci数列,且有 \( P_n = Q_{n+1}, n \geqslant 0 \)。
\end{solution}

\begin{exercise}
	\begin{enumerate}
		\item 计算以下有限连分数的值和各个渐近分数:

		      \noindent
		      \begin{enumerate*}[label={(\roman*)}, itemjoin={{,\hspace{3em}}}]
			      \item \( \langle 1,2,3 \rangle \)
			      \item \( \langle 0,1,2,3 \rangle \)
			      \item \( \langle 3,2,1 \rangle \)
			      \item \( \langle 2,1,1,4,1,1 \rangle \)
			      \item \( \langle -4,2,1,7,8 \rangle \)
			      \item \( \langle -1,1/2,1/3 \rangle \)
			      \item \( \langle 1/2,1/4,1/8,1/16 \rangle \)
		      \end{enumerate*}
		\item 把下面的有理数表为有限简单连分数,并求各个渐近分数:

		      \noindent
		      \begin{enumerate*}[label=(\roman*), itemjoin={{,\hspace{3em}}}, after={{.}}]
			      \item \( 121/21 \)
			      \item \( -19/29 \)
			      \item \( 177/292 \)
			      \item \( 873/4867 \)
		      \end{enumerate*}

		\item 求有限简单连分数 \( \langle 2,1,2,1,1,4,1,1,6,1,1,8 \rangle \)的各个渐近分数及其值。并与自然对数 \( e
		      \)的值比较。

		\item 设 \( a, b \)是正数。证明
		      \begin{equation*}
			      a + \sqrt{a^2 + b} = 2a + \dfrac{b}{2a + \dfrac{b}{2a + \dfrac{b}{a + \sqrt{a^2 + b}}}},
		      \end{equation*}
		      以及
		      \begin{equation*}
			      a + \sqrt{a^2 + b} = \langle 2a, 2a/b, 2a, 2a/b, 2a, 2a/b, a + \sqrt{a^2 + b} \rangle .
		      \end{equation*}

		\item 若 \( \xi_0 = \langle x_0, x_1, \cdots, x_n \rangle, x_0 > 0 \),则 \( \xi_0^{-1}= \langle 0, x_0, x_1,
		      \cdots, x_n \rangle. \)

		\item 设 \( {x_j},{P_j},{Q_j} \) 同\S \ref{sec:连分数介绍}定理\ref{thrm:PQ}。证明:
		      \begin{enumerate}[label=(\roman*)]
			      \item 当 \( n \geqslant 1\)时, \( Q_n/Q_{n-1} = \langle x_n, x_{n-1}, \cdots, x_1 \rangle. \)
			      \item 当 \( x_0 > 0, n \geqslant 0 \)时, 	\( P_n/P_{n-1} = \langle x_n, x_{n-1}, \cdots, x_1, x_0
			            \rangle. \)
		      \end{enumerate}
		\item 在\S\ref{sec:连分数介绍}定理\ref{thrm:PQ}的条件和符号下:证明:
		      \begin{enumerate}[label=(\roman*)]
			      \item 当 \( n \geqslant 1 \)时
			            \begin{align*}
				             & Q_{2n} \geqslant x_1(x_2 + x_4 + \cdots + x_{2n}) + 1, \\
				             & Q_{2n-1} \geqslant x_1 + x_3 + \cdots + x_{2n-1},      \\
				             & Q_n < (1 + x_1)(1+x_2)\cdots(1+x_n).
			            \end{align*}
			      \item 无限连分数 \( \langle x_0, x_1, x_2, \cdots \rangle \)收敛的就充要条件是级数 \( \displaystyle
			            \sum_{j=0}^{\infty}x_j \)发散。
			      \item \( \langle 1/2, 1/2, 1/2, \cdots \rangle = (\sqrt{17} + 1)/4. \)
			      \item \( \langle 4,1/4,4,1/4, \cdots \rangle = \sqrt{20} + 2. \)
		      \end{enumerate}

		\item 证明:
		      \begin{equation*}
			      \langle 0,x_1, \cdots, x_n \rangle = 1/(Q_0Q_1) - 1/(Q_1Q_2) + \cdots + (-1)^{n-1}/(Q_{n-1}Q_n).
		      \end{equation*}
		\item 证明:当 \( n \geqslant 0 \)时 \( Q_{n+1}(x_0, x_1, \cdots, x_{n+1}) = P_n(x_1, x_2, \cdots, x_{n+1}),
		      \)这里 \( P_n, Q_n \)同\S\ref{sec:连分数介绍}式\eqref{eq:递推函数形式}。
		\item 证明:
		      \begin{enumerate}[label=(\roman*)]
			      \item 当 \( n \geqslant 0 \)时,
			            \begin{equation*}
				            P_n =
				            \begin{vmatrix}
					            x_0 & -1  &        &        &         & 0   \\
					            1   & x_1 & -1     &        &         &     \\
					                & 1   & x_2    & \ddots &         &     \\
					                &     & \ddots & \ddots & \ddots  &     \\
					                &     &        & \ddots & x_{n-1} & -1  \\
					            0   &     &        &        & 1       & x_n
				            \end{vmatrix}.
			            \end{equation*}
			      \item 当 \( n \geqslant 1 \)时
			            \begin{equation*}
				            Q_n =
				            \begin{vmatrix}
					            x_1 & -1  &        &        &         & 0   \\
					            1   & x_2 & -1     &        &         &     \\
					                & 1   & x_3    & \ddots &         &     \\
					                &     & \ddots & \ddots & \ddots  &     \\
					                &     &        & \ddots & x_{n-1} & -1  \\
					            0   &     &        &        & 1       & x_n
				            \end{vmatrix}.
			            \end{equation*}
		      \end{enumerate}
		\item 证明:当 \( n \geqslant 1 \) 时
		      \begin{equation*}
			      \begin{pmatrix}
				      P_n & P_{n-1} \\
				      Q_n & Q_{n-1}
			      \end{pmatrix}
			      =
			      \begin{pmatrix}
				      x_0 & 1 \\
				      1   & 0
			      \end{pmatrix}
			      \begin{pmatrix}
				      x_1 & 1 \\
				      1   & 0
			      \end{pmatrix}
			      \cdots
			      \begin{pmatrix}
				      x_n & 1 \\
				      1   & 0
			      \end{pmatrix}.
		      \end{equation*}
		\item 设 \( a^{(j)} \)由\S\ref{sec:连分数介绍}式\eqref{eq:an标记}给出, \( P_j, Q_j
		      \)由\S\ref{sec:连分数介绍}定理\ref{thrm:PQ}给出。证明:当 \( 1 \leqslant j \leqslant n \)时,
		      \begin{equation*}
			      Q_j|Q_{j-1}a^{(n)}-P_{j-1}| + Q_{j-1}|Q_ja^{(n)} - P_j| = 1.
		      \end{equation*}

	\end{enumerate}
\end{exercise}

\section{有限简单连分数} \label{sec:有限简单连分数}
本节讨论有限简单连分数的性质及其与有理分数的关系。
设 \( a_0, a_1, a_2, \dots \)是一个无限整数列, \( a_j \geqslant 1, j \geqslant 1 \)。记有限简单连分数
\begin{equation}
	\langle a_0, a_1, \dots, a_n \rangle = r^{(n)}, \quad n \geqslant 0. \label{eq:有限简单连分数}
\end{equation}
定义整数列 \( \left\{ h_n\right\} \)与 	\( \left\{k_n\right\} \):
\begin{equation}\label{eq:hk定义}
	\begin{cases}
		h_{-2} = 0,\, h_{-1} = 1,\, k_{-2} - 1,\, k_{-1} = 0, \\
		h_n = a_nh_{n-1} + h_{n-2},\, k_n = a_nk_{n-1} + k_{n-2}, n \geqslant 0 .
	\end{cases}
\end{equation}
显见, \( h_0 = a_0,\, h_1 = a_1a_0 + 1 \),
\begin{equation}
	1 = k_0 \geqslant a_1 = k_1 < k_2 < \dots < k_n < \dots, \quad k_n \to + \infty .
\end{equation}
这里的 \( \left\{ a_j \right\} \), \(\left\{ h_n \right\}\), \(\left\{ k_n
\right\}\)就相当于\S\ref{sec:连分数介绍} 中的\eqref{thrm:PQ} \(\left\{ x_j \right\}\), \(\left\{ P_n \right\}\),
\(\left\{ Q_n \right\}\)取整数的特殊情形,作为\S\ref{sec:连分数介绍}的定理\ref{thrm:PQ}的特例就得到
\begin{theorem}
	有限简单连分数\eqref{eq:有限简单连分数}的值是有理分数,
	\begin{equation}
		r^{(n)} = \langle a_0, \dots, a_n \rangle = h_n/k_n,\, \left(h_n, k_n\right) = 1, n \geqslant 0,
	\end{equation}
	其中 \( h_n, k_n \)由式\eqref{eq:hk定义}给出。此外,还有
	\begin{align}
		h_nk_{n-1} - h_{n-1}k_n = (-1)^{n+1}, \qquad               & n \geqslant -1,           \\
		h_nk_{n-2} - h_{n-2}k_n = (-1)^{n}a_n, \qquad              & n \geqslant \phantom{-}0, \\
		r^{(n)} - r^{(n-1)} = (-1)^{n+1}(k_nk_{n-1})^{-1}, \qquad  & n \geqslant \phantom{-}1,
		\label{eq:有限简单连分数误差1}                                                         \\
		r^{(n)} - r^{(n-2)} = (-1)^{n}a_n(k_nk_{n-2})^{-1}, \qquad & n \geqslant \phantom{-}2.
		\label{eq:有限简单连分数误差2}
	\end{align}
	对于给定的一个不是整数的 有理分数 \( u_0 / u_1, u_1 \geqslant 2 \),如何来得到它的有限简单连分数表示式呢?\S
	\ref{sec:连分数介绍}的例\ref{ex:化简分数}实际上已 经给出 了这种方法。利用第一章§5的辗转相除法可得:
	\begin{equation}\label{eq:有限简单连分数辗转相除}
		\begin{cases}
			u_0 = b_0 u_1 + u_2, \hfill              0 < u_2 < u_1,     \\
			u_1 = b_1 u_2 + u_3, \hfill              0 < u_3 < u_2,     \\
			\begin{minipage}{0.4\textwidth}
				\dotfill
			\end{minipage}                              \\
			u_{s-1} = b_{s-1} u_s + u_{s+1}, \hfill  0 < u_{s+1} < u_s, \\
			u_s = b_s u_{s+1}. \hfill
		\end{cases}
	\end{equation}
	由于 \( u_0 / u_1 \)不是整数,所以 \( s \geqslant 1 \),以及 \( b_s \geqslant 2 \)。 设
	\begin{equation}
		\xi_j = u_j / u_{j+1}, \qquad 0 \leqslant j \leqslant s.
	\end{equation}
	由式\eqref{eq:有限简单连分数辗转相除}得
	\begin{equation}
		\begin{cases}
			\xi_j = b_j + 1 / \xi_{j + 1} = \langle b_j, \xi_{j+1} \rangle, \qquad 0 \leqslant j \leqslant s \\
			\xi_s = b_s.
		\end{cases}
	\end{equation}
	利用\S \ref{sec:连分数介绍}式\eqref{thrm01eq:01}得
	\begin{align}
		\xi_0 & = u_0 / u_1 = \langle b_0, \xi_1 \rangle = \langle b_0, \langle b_1, \xi_2 \rangle \rangle \nonumber \\
		      & = \langle b_0, b_1, \xi_2 \rangle = \langle b_0, b_1, \langle b_2, \xi_3 \rangle \rangle   \nonumber \\
		      & = \langle b_0, b_1, b_2, \xi_3 \rangle = \dots \nonumber                                             \\
		      & = \langle b_0, b_1, \dots, b_{s-1}, \xi_s \rangle \nonumber                                          \\
		      & = \langle b_0, b_1, \dots, b_s \rangle, \qquad b_s > 1. \label{eq:有限简单连分数表达式1}
	\end{align}
	这就得到了 \( \xi_0 = u_0 / u_1 \)的有限简单连分数表示式。由于 \( b_s \geqslant 2 \),由\S \ref{sec:连分数介绍}得
	\begin{align}
		\xi_0 & = u_0 / u_1 = \langle b_0, b_1, \dots, (b_s - 1) + 1/1 \rangle\nonumber                  \\
		      & = \langle b_0, b_1, \dots, b_{s-1}, b_s - 1, 1 \rangle. \label{eq:有限简单连分数表达式2}
	\end{align}
	这样,有理分数 \( u_0 / u_1
	\)就有两个有限简单连分数表示式\eqref{eq:有限简单连分数表达式1}和\eqref{eq:有限简单连分数表达式2},\eqref{eq:有限简单连分数表达式1}的最后一个部分商是
	\( b_s \geqslant 2
	\),\eqref{eq:有限简单连分数表达式2}的最后一个部分商为1。那么,是否会有别的的形式的表示式呢?回答是否定的。这是下面的唯一性定理。
\end{theorem}

\begin{theorem}\label{thrm:简单连分数相等}
	设 	\( \langle a_0, \cdots, a_n \rangle, \langle b_0, \cdots, b_s \rangle \)是两个简单连分数, \( a_n > 1, b_s > 1
	\)。若
	\begin{equation}
		\langle a_0, \cdots, a_n \rangle = \langle b_0, \cdots, b_s \rangle,
		\label{eq:简单连分数相等}
	\end{equation}
	则必有 \( s = n, a_j = b_j, 0 \leqslant j \leqslant n \)。
\end{theorem}
\begin{proof}
	不妨设 \( s \geqslant n \)。对 \( n \)用归纳法。当 \( n=0 \)时,若 \( s \geqslant 1
	\),则由\S\ref{sec:连分数介绍}式\eqref{thrm01eq:01}得
	\begin{align*}
		a_0 & = \langle b_0, b_1, \cdots, b_s \rangle = \langle b_0, \langle b_1, \cdots, b_s \rangle \rangle \\
		    & = b_0 + 1 / \langle b_1, \cdots, b_s \rangle.
	\end{align*}
	由于 \( b_s > 1 \),所以 \( \langle b_1, \cdots, b_s \rangle > 1\),因此上式不可能成立。这就推出 \( s=0, a_0=b_0
	\)。所以结论当 \( n=0 \)时成立。假设当 \( n=k(\geqslant 0) \)时结论成立。当 \( n=k+1
	\)时,由\S\ref{sec:连分数介绍}式\eqref{thrm01eq:01}得(注意 \( s \geqslant n \geqslant 1 \))
	\begin{gather*}
		\langle a_0, a_1, \cdots, a_{k+1}\rangle = a_0 + 1/ \langle a_1, \cdots, a_{k+1} \rangle. \\
		\langle b_0, b_1, \cdots, b_s \rangle = b_0 + 1/ \langle b_1, \cdots, b_s \rangle.
	\end{gather*}
	由 \( a_{k+1} > 1 \)及 \( b_s > 1 \)知 \( \langle a_1, \cdots, a_{k+1} \rangle > 1 \)及 \( \langle b_1, \cdots, b_s
	\rangle > 1 \)。因而,由条件\eqref{eq:简单连分数相等} \(  n=k+1  \),就推出 \( a_0 = b_0 \)及
	\begin{equation*}
		\langle a_1, \cdots, a_{k+1} \rangle = \langle b_1, \cdots, b_s \rangle.
	\end{equation*}
	由归纳假设知,从上式就推得 \( s = k + 1 \)及 \( a_j = b_j \), \( 1 \geqslant j \geqslant k + 1 \)。这就证明了当 \(
	n = k + 1\)时结论也成立。所以结论对一切 \( n \geqslant 0 \)都成立。
\end{proof}

(由定理\ref{thrm:简单连分数相等}及式\eqref{eq:有限简单连分数表达式1}就立即推出:
\begin{theorem}
	任一不是整数的有理分数 \( u_0 / u_1 (\)有且仅有式\eqref{eq:有限简单连分数表达式1}及\eqref{eq:有限简单连分数表达式2}给出的两种有限简单连分数表示式,其中 \( b_0, \cdots, b_s
	\)由 \eqref{eq:有限简单连分数辗转相除}给出, \( s \geqslant 1, b_s > 1 \)。
\end{theorem}

\begin{example}
	求 \( 13/5 \)的有限简单连分数。
\end{example}
\begin{solution}
	我们按式\eqref{eq:有限简单连分数表达式1}来求。
	\begin{align*}
		13/5 & = \langle 2 + 3/5 \rangle = \langle 2, 5/3 \rangle = \langle 2, \langle 1+2/3 \rangle \rangle \\
		     & = \langle 2, \langle 1, 3/2 \rangle \rangle = \langle 2, 1, 1 + 1/2 \rangle                   \\
		     & = \langle 2, 1, 1, 2 \rangle.
	\end{align*}
	这个有限简单连分数的特点是:从左往右和从右往左的数字是一样的(参见习题二第5题)。 \( 13/5 \)还可表为
	\begin{equation*}
		13/5 = \langle 2, 1, 1, 1, 1 \rangle.
	\end{equation*}
\end{solution}

\begin{example}
	求 \( 7700/2145 \)的有限简单连分数,及它的各个渐近分数。
\end{example}

\begin{solution}
	\begin{align*}
		7700/2145 & = \langle 3 + 1265/2145 \rangle = \langle 3, 2145/1265 \rangle               \\
		          & = \langle 3, 1 + 880/1265 \rangle = \langle 3, 1, 1265/880 \rangle           \\
		          & = \langle 3, 1, 1 + 385/880 \rangle = \langle 3, 1, 1, 880/385 \rangle       \\
		          & = \langle 3, 1, 1, 2 + 110/385 \rangle = \langle 3, 1, 1, 2, 385/110 \rangle \\
		          & = \langle 3, 1, 1, 2, 3 + 55/110 \rangle = \langle 3, 1, 1, 2, 3, 2 \rangle  \\
		          & = \langle 3, 1, 1, 2, 3, 1, 1 \rangle
	\end{align*}
	按\S\ref{sec:有限简单连分数}式\eqref{eq:hk定义}列表来求 \( h_n, k_n \)(见表\ref{tab:求hnkn})及渐近分数 \( h_n/k_n
	\)。由表\ref{tab:求hnkn}知
	\begin{table}[htbp]
		\centering
		\begin{tabularx}{\textwidth}{XXXXXXX}\toprule
			\( n \)   & 0 & 1 & 2 & 3  & 4  & 5   \\ \midrule
			\( a_n \) & 3 & 1 & 1 & 2  & 3  & 2   \\
			\( h_n \) & 3 & 4 & 7 & 18 & 61 & 140 \\
			\( k_n \) & 1 & 1 & 2 & 5  & 17 & 39  \\\bottomrule
		\end{tabularx}
		\caption{求 \( h_n, k_n \)}
		\label{tab:求hnkn}
	\end{table}

	渐近分数依次为
	\begin{equation*}
		3/1, 4/1, 7/2, 18/5, 61/17, 140/39 = 7700/2145。
	\end{equation*}

	这分数值化简后为 \( 140/39
	\)。虽然原来的分数不是既约的,但渐近分数一定是既约的。利用式\eqref{eq:有限简单连分数误差1}和\eqref{eq:有限简单连分数误差2}可以计算原分数与渐近分数的误差,如
	\begin{align*}
		140/39 - 61/17 & = h_5/k_5 - h_4/k_4 = 1/(k_4k_5)    \\
		               & = 1 / (17 \cdot 39) = 1 / 663,      \\
		140/39 - 18/5  & = h_5/k_5 - h_3/k_3 = -a_5/(k_3k_5) \\
		               & = -2/(5\cdot39) = -2/195.
	\end{align*}
	其它几个误差请读者自己计算。
\end{solution}

\begin{exercise}
	\begin{enumerate}
		\item 设 \( a/b \) 是有理分数, \( a_0, \cdots, a_n \)是它的有限简单连分数, 以及 \( b \geqslant 1 \)。证明:
		      \begin{equation*}
			      ak_{n-1} - bh_{n-1} = (-1)^{n+1}(a,b)。
		      \end{equation*}
		\item 具体说明第1题给出了求最大公约数 \( (a,b) \)及解不定方程 \( ax+by=c \)的一个新方法.用这个方法来求解以下的最大公约
		      数和不定方程。
		      \begin{enumerate}[label=(\roman*)]
			      \item \( 205x+93y=1 \);
			      \item  \( 65x-56y= -1 \);
			      \item \( 13x+17y=5 \);
			      \item \( 77x+63y=40 \);
			      \item \( (314,159)\);
			      \item  \( (4144,7696) \)。
		      \end{enumerate}
		\item 求有理分数
		      \begin{enumerate*}[label=(\roman*), itemjoin={,\hspace{1em}}]
			      \item \( 7/11 \)
			      \item \( 173/55 \)
			      \item \( -43/1001 \)
			      \item \( 5391/3976 \)
			      \item \( -873/4867 \)
		      \end{enumerate*}
		      的两种有限简单连分数表示式,以及它们的各个渐近分数、渐近分数与有理分数的误差.
		\item 设 \( r_0 = \langle a_0, \cdots, a_n \rangle, s_0 = \langle b_0, \cdots, b_n, b_{n+1} \rangle
		      \)是两个有限简单连分数。试求:
		      \begin{enumerate*}[label=(\roman*)]
			      \item \( r_0 = s_0 \)的充要条件;
			      \item \( r_0 < s_0 \)的充要条件。
		      \end{enumerate*}
		\item 设有理分数 \( a/b((a,b) = 1, a\geqslant b \geqslant 1) \)的有限简单连分数是 \( a_0, a_1, \cdots, a_n
		      \)。证明:
		      \begin{equation*}
			      \langle a_0, a_1, \cdots, a_{n-1}, a_n \rangle = \langle a_n, a_{n-1}, \cdots, a_1, a_0 \rangle
		      \end{equation*}
		      的充要条件是
		      \begin{enumerate*}[label=(\roman*)]
			      \item 当\( 2 \nmid n \)时, \( a \mid b^2 + 1 \);
			      \item 当\( 2 \mid n  \)时, \( a \mid b^2 - 1 \)。
		      \end{enumerate*}
		\item 设 \( a, b, c, d \)是整数, \( c > d > 0, \quad ad - bc = \pm 1 \)。再设实数 \( \eta \geqslant 1 \)。若
		      \( \xi = (a\eta + b) / (c\eta + d) \),则 \( \xi = \langle a_0, \cdots, a_n, \eta \rangle \),以及 \( b/d
		      = \langle a_0, \cdots, a_{n-1} \rangle\),这里 \( \langle a_0, \cdots, a_n \rangle \)是 \( a/c \)的有限简连分数表示式。
		  \item 设 \( r^{(n)} = \langle 1, 2, \cdots, n+1 \rangle \),它的各个渐近分数为 \( h_j/k_j \),
		      \( j = 0, 1, \cdots, n \)。证明:当\( n \geqslant 3 \)时,
			  \begin{equation*}
				  h_n = nh_{n-1} + nh_{n-2} + (n-1)h_{n-3} + \cdots + 3h_1 + 2h_0 + 2h_{-1}。
			  \end{equation*}

	\end{enumerate}
\end{exercise}
